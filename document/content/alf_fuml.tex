\chapter{Einführung in ALF und fUML}

\textbf{Motivation}
\begin{itemize}
\item E-Commerce Ordering System
\item Designed in UML
\item Implementiert im Web (Programmiersprachen unabhängig
\end{itemize}

\section{UML-Model: Ordering}

\img{0.75}{document/graphics/ordering-model.png}{Bestellung Klassen-Diagramm}{orderingmodel}

In Abbildung \imgref{orderingmodel} wird ein UML-Diagramm verwaltung von Bestellungen dargestellt. Die Klasse Order ist Verantwortlich für die Sortierung.

\textbf{Anmerkung zu UML:}
Der Punkt beim Pfeil: Das OrderLineItem kann direkt auf das Produkt zugreifen aber umgekehrt nicht. Im OrderLineItem ist ein Property product aber im Product ist keine Liste von OrderLineItem. Nur ein Pfeil ist was ähnliches aber nicht so strikt.

\section{State-Diagramm: Ordering}

\img{0.75}{document/graphics/ordering-state.png}{Bestellung State-Diagramm}{orderingstate}

In Abbildung \imgref{orderingstate} ist das Zustandsdiagramm einer Bestellung dargestellt. Die Schleife bei 2 und 3 besagt, dass der User aufgefordert wird die Zahlungsmöglichkeit auszuwählen. Fals das ncht klappt wird noch einmal gefragt. Es fehlt hier eine Abbruchbedinung.

\section{Executable UML Foundation (fUML)}

Foundational UML (fUML) is an executable subset of standard UML that can be used to define, in an operational style, the structural and behavioral semantics of systems.

\textbf{Ziele:}
\begin{itemize}
\item UML ist nicht genugt spezifiziert um ausführbar zu sein
\item fUML präzisiert ein ausführbares subset von UML
\item Graphiches Modellierung ist nicht genug programmierung
\item ALF spezifiziert eine Action Language für UML
\end{itemize}

\subsection{Key Komponenten}

Foundational UML Subset (fUML) – A computationally complete subset of the abstract syntax of UML (Version 2.3)
\begin{itemize}
\item Kernel–Basicobject-orientedcapabilities
\item Common Behavior – General behavior and asynchronous communication
\item Activities – Activity modeling, including structured activities (but not including variables, exceptions, swimlanes, streaming or other “higher level” activity modeling)
\end{itemize}

Execution Model – A model of the execution semantics of user models within the fUML subset

Foundational Model Library
\begin{itemize}
\item Primitive Types – Boolean, String, Integer, Unlimited Natural
\item Primitive Behaviors – Boolean, String and Arithmetic Functions
\item Basic Input/Output – Based on the concept of “Channels”
\end{itemize}

\section{UML Action Language (Alf)}

The Action Language for Foundational UML (Alf) is a textual surface representation for UML modeling elements with the primary purpose of acting as the surface notation for specifying executable (fUML) behaviors within an overall graphical UML model. (But which also provides an extended notation for structural modeling within the fUML subset.)

\subsection{Key Komponenten}

\paragraph{Concrete Syntax} – A BNF specification of the legal textual syntax of the Alf language.

\paragraph{Abstract Syntax} – A MOF metamodel of the abstract syntax tree that is synthesized during parsing of an Alf text, with additional derived attributes and constraints that specify the static semantic analysis of that text.

\paragraph{Semantics} – The semantics of Alf are defined by mapping the Alf abstract syntax metamodel to the fUML abstract syntax metamodel.

\paragraph{Standard Model Library}
\begin{itemize}
\item From the fUML Foundational Model Library
\end{itemize}

\paragraph{Primitive Types} (plus Natural and Bit String)

\paragraph{Primitive Behaviors} (plus Bit String Functions and Sequence Functions)

\paragraph{Basic Input/Output}
\begin{itemize}
\item Collection Functions – Similar to OCL collection operations for sequences
\item Collection Classes – Set, Ordered Set, Bag (Stack), List, Queue, Deque, Map
\end{itemize}

\subsection{Elements of Executable UML}

\img{0.75}{document/graphics/example-exec-uml.png}{Beispiel Ausführbares UML}{execuml}

Abbildung \imgref{execuml} zeigt ein Beispiel für ein Ausführbares UML.

\section{Actions}

\subsection{Invocation Actions}
Beinhaltet Aufrufen von Operationen, Verhalten in gewissen Situationen, das Senden von Signalen, und akzeptieren von Events.

\subsection{Object Actions}
Unter diesem Punkt werden Aktionen auf Objekte zusammengefasst. Dazu gehören Spezifikationen, das Erzeugen und zestören von Objekten. Mit instanceof können Typen von Variablen überprüft werden, wobei auch die Vererbungshierarchie berücksichtigt wird (im Gegensatz zu C++, wo deshalb so viel mit Pointern gearbeitet wird).

Mit reclassify kann man auch dynamisch den Typ von Objekten ändern. Damit kann man z.B. auch Zustände mit Klassen abbilden, und das als Typ des Objekts verwenden.

\subsection{Structural Feature Actions}
Bei dieser Gruppe geht es um das Setzen und Lesen von Features bzw. Eigenschaften der einzelnen Elemente. Dabei werden auch Listenfunktionen beachtet.

\subsection{Link Actions}
Beinhaltet Lesen, Erzeugen, Zerstören und das Zurücksetzen von Assoziationen.
\subsection{Computation}
Dazu gehört das Indizieren (wichtig: startet mit 1), Vergleichen, Ausführen von arithmethischen Operationen, Inkrementierung und Dekrementierung.

\section{Structures}

\subsection{Classes and attributes}
classes-attributes.png

\subsection{Datatypes}
In UML dürfen Datentypen nur Attribute haben, aber keine Operationen, was einem Grundsatz der Objektorientierung widerspricht.

Verfügbare Datentypen in UML sind Boolean, Integer, UnlimitedNatural und String, mit fUML kommt auch noch der Datentyp Real dazu.

\subsection{Operations and Methods}
methods.png

\subsection{Structural Semantics}
Mit diesen Diagrammen kann man Objektstrukturen darstellen, und kann visualisieren wie die definierten Klassendiagramme umgesetzt werden.

\subsection {Behavioral Semantics}
Zeigt wie sich der Aufruf einer Funktoin auf die Objektstruktur auswirkt. Folgende Abbildung zeigt ein Beispiel. In diesen Diagrammen werden auch die Auswirkungen der Modellierung dargestellt.

cancel-order.png

\section{Asynchronous Communication}

\subsection{Signals and Receptions}
Signale können auch Attribute und somit komplexe Strukturen haben. Eine Reception auf einem anderen Objekt definiert, dass dieses Objekt dieses Signal empfangen und verarbeiten kann.

\section{Primitive Behaviours}
Es werden auch primitive Behaviour von Datentypen mit ausgeliefert, wie Vergleiche, Umwandlungen und arithmetische Operationen bei numerischen Datentypen, oder Konkatenierung oder das zurückliefern von Teilstrings bei String.

\section{Collections}
Dabei handelt es sich um eine Ansammlung von mehreren Objekten oder Ausprägungen primitiver Datentypen. Kennt man auch aus anderen Programmiersprachen, etwas verwirrend ist nur der Begriff Deque, der im Prinzip einem Stack entspricht.
